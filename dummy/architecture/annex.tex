\subsection{\informed\ Main Program}

This section show you how an \informed\ Main Program is represented pictorially on an architecture
diagram, and as a SPARK package.

\begin{figure}[h]
  \begin{center}
    \begin{tikzpicture}[x=2cm,y=2cm,shorten >=2pt, shorten <=2pt]
      \node[main] (main) at (0, 0) {main};
    \end{tikzpicture}
  \end{center}
  \caption{\informed\ Main Image Drawing Element}
\end{figure}

\subsection{\informed\ Variable Packages}

This section show you how an \informed\ Variable Package is represented pictorially on an architecture
diagram, and as a SPARK package.

\begin{figure}[h]
  \begin{center}
    \begin{tikzpicture}[x=2cm,y=2cm,shorten >=2pt, shorten <=2pt]
      \node[var] (variable_package)                  at (0, 0) {Variable};
    \end{tikzpicture}
  \end{center}
  \caption{\informed\ Variable Package Drawing Element}
\end{figure}


\begin{lstlisting} [label=some-code,caption=Some Code]

with Time;

package Clock
  with Abstract_State => State
is

  procedure Initialise_To_Zero
    with Global => (Output => State);
  --  S.reset

  procedure Increment
    with Global => (In_Out => State);
  --  S.count

  function Get_Current_Time return Time.T
    with Global => (Input => State);
  --  S.display

end Clock;
\end{lstlisting}


\subsection{\informed\ Type Packages}

This section show you how an \informed\ Type Package is represented pictorially on an architecture
diagram, and as a SPARK package.

\begin{figure}[h]
  \begin{center}
    \begin{tikzpicture}[x=2cm,y=2cm,shorten >=2pt, shorten <=2pt]
      \node[type]     (type_package)                 at ( 0, 0) {Type};
    \end{tikzpicture}
  \end{center}
  \caption{\informed\ Type Package Drawing Element}
\end{figure}


subsection{\informed\ Utility Packages}

This section show you how an \informed\ Utility Package is represented pictorially on an architecture
diagram, and as a SPARK package.

 \begin{figure}[h]
  \begin{center}
    \begin{tikzpicture}[x=2cm,y=2cm,shorten >=2pt, shorten <=2pt]
      \node[utility] (utility) at (0, 0) {Utility};
    \end{tikzpicture}
  \end{center}
  \caption{\informed\ diagram of our Stopwatch}
\end{figure}


\subsection{\informed\ Boundary Variable Packages}

This section show you how an \informed\ Boundary Variable Package is represented pictorially on an architecture
diagram, and as a SPARK package.

\begin{figure}[h]
  \begin{center}
    \begin{tikzpicture}[x=2cm,y=2cm,shorten >=2pt, shorten <=2pt]
      \node[boundary] (boundary_package)       at (0, 0) {Boundary};
    \end{tikzpicture}
  \end{center}
  \caption{\informed\ diagram of our Stopwatch}
\end{figure}

\subsection{\informed\ Connectors}
