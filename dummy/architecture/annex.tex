\subsection{\informed\ Main Program}

\noindent\parbox[][][t]{.2\linewidth}{
    \begin{tikzpicture}[x=2cm,y=2cm,shorten >=2pt, shorten <=2pt]
      \node[main] (main) at (0, 0) {main};
    \end{tikzpicture}}%
 \parbox[][][t]{.8\linewidth}{
This section show you how an \informed\ Main Program is represented pictorially on an architecture
diagram, and as a SPARK package.

Typically, the SPARK implementation of an \informed\ design will have only one
such main program (annotated with SPARK's \texttt{main\textunderscore program} annotation).
}

\begin{lstlisting} [label=some-code,caption=An \informed\ SPARK Main program]
with A, B, C, D;

procedure Main
  with Global => (Input => (I1,
                            I2,
                            I3),
                  Output => (O1,
                             O2))
is
begin  -- Main
   Initialize;
   loop
      ControlProcedure;
   end loop;
end Main;
\end{lstlisting}


\subsection{\informed\ Variable Packages}

\noindent\parbox[][][t]{.2\linewidth}{
    \begin{tikzpicture}[x=2cm,y=2cm,shorten >=2pt, shorten <=2pt]
      \node[var] (variable_package)                  at (0, 0) {Variable};
    \end{tikzpicture}}%
 \parbox[][][t]{.8\linewidth}{

This section show you how an \informed\ Variable Package is represented pictorially on an architecture
diagram, and as a SPARK package.}

Rather than provide a generic example, this section uses a Stack package to provides a concrete example of an
\informed\ variable package. In this example we are defining a container (\texttt{Stack.State}) that can contain
values of a certain shape ("abstract stack") upon which certain operations can be performed (\texttt{Clear},
\texttt{Push} and \texttt{Pop}).


\begin{lstlisting} [label=some-code,caption=An \informed\ SPARK variable package]
package Stack
  with Abstract_State => State
is

  procedure Clear;
    with Global => (Output => State);

  procedure Push(X : in Integer);
    with Global => (In_Out => State);

  procedure Pop(X : out Integer);
    with Global => (In_Out => State);

end Stack;
\end{lstlisting}


\subsection{\informed\ Type Packages}

\noindent\parbox[][][t]{.2\linewidth}{
    \begin{tikzpicture}[x=3cm,y=3cm,shorten >=2pt, shorten <=2pt]
      \node[type]     (type_package)                 at ( 0, 0) {Type};
    \end{tikzpicture}}%
 \parbox[][][t]{.8\linewidth}{

This section show you how an \informed\ Type Package is represented pictorially on an architecture
diagram, and as a SPARK package. This type of package does not have a state and therefore does not
require a State declaration, and this is shown in the code below.
}

The stack variable package presented earlier can be restructured as a type package declaring an
abstract type.

\begin{lstlisting} [label=some-code,caption=An \informed\ SPARK Type Package]
package Stack is

  type T is private;

  procedure Clear;
    with Global => (Output => State);

  procedure Push(X : in Integer);
    with Global => (In_Out => State);

  procedure Pop(X : out Integer);
    with Global => (In_Out => State);

  private
    --#hide stack;
    -- full declaration of type T would go here
end Stack;
\end{lstlisting}


\subsection{\informed\ Utility Packages}
\noindent\parbox[][][t]{.2\linewidth}{
    \begin{tikzpicture}[x=2cm,y=2cm,shorten >=2pt, shorten <=2pt]
      \node[utility] (utility) at (0, 0) {Utility};
    \end{tikzpicture}}%
 \parbox[][][t]{.8\linewidth}{

This section show you how an \informed\ Utility Package is represented pictorially on an architecture
diagram. No example of a SPARK Utility package is provided here since these are of such a general nature.
}

However, care should be taken when implementing an \informed\ design as it is easy for there to be
an excessive proliferation of packages. In many cases the correct place for an operation is in the
type or variable pacakage upon which it operates. The key thing for an \informed\ design is for the
operations to be correctly located.

\subsection{\informed\ Boundary Variable Packages}
\noindent\parbox[][][t]{.2\linewidth}{
    \begin{tikzpicture}[x=2cm,y=2cm,shorten >=2pt, shorten <=2pt]
      \node[boundary,text width=1cm] at (0, 0) {Boundary};
     \end{tikzpicture}}%
 \parbox[][][t]{.8\linewidth}{

This section show you how an \informed\ Boundary Variable Package is represented pictorially on an architecture
diagram, and as a SPARK package.
}
A boundary variable is a variable package; however, unlike other instances of such variables the name
provided in its own variable clause is a place holder representing the stream of data arriving from,
or being sent tom the outside world rather than simply an abstract name for the internal state of the
package.

It is often useful to place an abstraction layer between the boundary variables of a system and their uesrs;
this approach is appropriate where direct use of the boundary variables would provide insufficent abstraction
allowing too much detail to become visable in higher level SPARK annotations. An example of this abstraction
layer is include in the code example shown below.

\begin{lstlisting} [label=some-code,caption=An \informed\ SPARK Boundary Variable Package]
--  Workaround, because we don't have volatile ins yet...

package Controls
  with Abstract_State => (Start, Stop, Reset)
is

  procedure Read_Start_State (Is_Pressed : out Boolean)
    with Global => (Input => Start);

  procedure Read_Stop_State (Is_Pressed : out Boolean)
    with Global => (Input => Stop);

  procedure Read_Reset_State (Is_Pressed : out Boolean)
    with Global => (Input => Reset);

end Controls;

--#inherit Controls;
private package Controls.Reset
   with Abstract_State => (Reset)
is
  procedure Read (Is_Pressed : out Boolean)
    with Global => (Input => Reset);
end Controls.Reset;

--#inherit Controls;
private package Controls.Start
   with Abstract_State => (Start)
is
  procedure Read (Is_Pressed : out Boolean)
    with Global => (Input => Start);
end Controls.Start;

--#inherit Controls;
private package Controls.Stop
   with Abstract_State => (Stop)
is
  procedure Read (Is_Pressed : out Boolean)
    with Global => (Input => Stop);
end Controls.Stop;

\end{lstlisting}

\subsection{\informed\ Connectors}
\noindent\parbox[][][t]{.2\linewidth}{
  \begin{tikzpicture}[x=2cm,y=2cm,shorten >=2pt, shorten <=2pt]
    \draw[->,strong] (0, 0) -- (1, 0);
    \draw[->,weak]   (0, 1) -- (1, 1);
  \end{tikzpicture}}%
\parbox[][][t]{.8\linewidth}{
Descriptive Text}
