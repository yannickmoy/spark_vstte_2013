\label{REVEALOverview}
Software Engineering is concerned with the controlled and correct
development of Systems (Machines) that impact upon the environment in
which they are deployed (World). Where the objective of the deployment
of a Machine is to improve the World.

The purpose of Requirements Engineering is to take a set of
Stakeholder Requirements and arrive at a complete, correct and usable
set of System Specification Statements for the system under
development. Within \reveal\ the application of the
process is also intended to have a wider objective of adding benefits
to the project Stakeholders through learning, negotiation and the
development of trust between all parties involved in the production of
the system.

\reveal\ is Altrans' systematic method for the
elicitation, specification and management of System
Requirements. Through its application we clearly identify three key
artefacts:

\begin{itemize}
\item Requirements (R) which are statements about things in the World
  that we want the System to make true.

\item Specification Statements (S) describe the Systems external
  behaviour.
  \begin{itemize}
  \item Specifications include only shared phenomena in the interface
    between the World and the System being developed.
  \item Specifications can only constrain shared phenomena that the
    System can control.
  \end{itemize}

\item Domain Statements (D) which are knowledge about the World, that
  is generally assumed and not normally documented, but must hold if
  the Specification Statements are to fully satisfy the System
  Requirements.
\end{itemize}

\noindent
Once this information has been gathered, a Satisfaction Argument is
constructed to show that all the Stakeholder Requirements have been
addressed through the System Specification and Domain Statements. (As
Verification and Validation progresses, V\&V Plans and summary results
can also inform the satisfaction argument, although this may not be
done for the competition.)

The \reveal\ process is far more extensive than that described above
including detailed processes for:

\begin{itemize}
\item Stakeholder Identification,
\item Knowledge Elicitation,
\item Requirement Verification and Validation,
\item Conflict Management and Resolution,
\item Requirements Maintenance and Management.
\end{itemize}

\noindent
More information on the \reveal\ method can be found on:
\begin{center}
  \scriptsize
  \url{http://intelligent-systems.altran.com/technologies/systems-engineering/revealtm.html}
\end{center}

\noindent
Since by its nature the VSTTE 2013 competition has a very short
timescale, and the Requirements are provided by the competition
organisers it is not expected that there would be time to apply a full
\reveal\ process, however the benefits of Requirements
capture, expression, satisfaction arguments and traceability can be
shown by their application in this context.

