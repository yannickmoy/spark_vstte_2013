\label{INFORMEDOverview}
\informed\ is the Altran method for the construction of high-quality
software at a low-cost, it uses elements of both object oriented (OOD)
and functional design. OOD is used to establish the architecture of
the system and the elements of system state it contains. The result is
an annotated framework of SPARK packages, constructing an architecture
that is modifiable and maintainable, which is capable of being
analysed at an early stage using the Examiner.

Altrans' analysis of many software designs has identified that there
are 3 key building blocks that are used repeatedly to build an
application. These building blocks are:

\begin{itemize}
\item A Main Program, that is the top level, entry point controlling
  the behaviour a system or sub-system.
\item Variable Packages, which are SPARK packages containing
  persistent "state", and are equivalent to what is commonly called an
  object.
\item Type Packages, which define the name of a type, and the
  operations taht it supports.
\end{itemize}

\noindent
In addition to these packages, \informed\ also includes two other
types of packages:

\begin{itemize}
\item Utility Packages, which provide shared functionality. The need
  for utility packages arises when an operation is required which
  affects or uses more than one variable of type package.
\item Boundary Variable Packages, which are particular kinds of
  variable package which provides interfaces between the software
  functionality described by the \informed\ design and elements
  outside it with which it must communicate. Unlike other variable
  packages the variable within the package is a place holder
  representing the stream of data arriving from, or being sent to, the
  outside world rather than simply an abstract name for the internal
  state of the package.
\end{itemize}

\noindent
The \informed\ method consists of 6 steps, and these are:

\begin{enumerate}
\item Identification of the System boundary, inputs and outputs.
\item Identification of the SPARK boundary.
\item Identification and localization of the system state.
\item Handling initalization of state.
\item Handling secondary requirements.
\item Implementing the internal behaviour of components.
\end{enumerate}

\noindent
The output of the \informed\ process is represented a collection of
design diagrams and a collection of well defined SPARK
specifications. A summary of the \informed\ graphical elements and
example representations are documented in Annex A of this document to
aid the reader in the understanding of the architectural design.

As the VSTTE 2013 competition has a very short timescale, it is not
expected that there is enough time to fully apply the \informed\
process, however the benefits of the benifits of the modelling
approach and the translation in SPARK Package Specifications can be
shown in this project.
